% !TeX root = ./pf2022.tex

%\includeonlyframes{current}

\section{Templates}

\begin{frame}[fragile]{Class template}

  \begin{itemize}
  \item<1-> Let's consider again the \code{Complex} class
  \item<2-> What if we want \code{float} members?
  \end{itemize}

  \begin{columns}[t]

  \begin{column}{.5\textwidth}
    \begin{codeblock}<1->{
class Complex \{
  \alert{double} r;
  \alert{double} i;
 public:
  Complex(
      \alert{double} x = \alert{double}\{\}
    , \alert{double} y = \alert{double}\{\}
  ) : r\{x\}, i\{y\} \{\}
  \alert{double} real() const \{ return r; \}
  \alert{double} imag() const \{ return i; \}
\};}\end{codeblock}

  \end{column}
  \begin{column}{.5\textwidth}
    \begin{codeblock}<2->{
class Complex \{
  \alert{float} r;
  \alert{float} i;
 public:
  Complex(
      \alert{float} x = \alert{float}\{\}
    , \alert{float} y = \alert{float}\{\}
  ) : r\{x\}, i\{y\} \{\}
  \alert{float} real() const \{ return r; \}
  \alert{float} imag() const \{ return i; \}
\};}\end{codeblock}

  \end{column}

  \end{columns}

\end{frame}

\begin{frame}[fragile]{Class template \insertcontinuationtext}

  We can transform \code{Complex} into a \textit{template}, with the
  floating-point type a parameter of that template

  \begin{codeblock}{
template<typename \alert{FP}> // or, template<class FP>
class Complex \{
  \uncover<9->{static_assert(std::is_floating_point_v<FP>); // (*)}
  \alert{FP} r;
  \alert{FP} i;
 public:
  Complex(\alert{FP} x = \alert{FP}\{\}, \alert{FP} y = \alert{FP}\{\}) : r\{x\}, i\{y\} \{\}
  \alert{FP} real() const \{ return r; \}
  \alert{FP} imag() const \{ return i; \}
\};

\uncover<2->{Complex c;                     // error}
\uncover<3->{Complex<double> d;             // instantiation of a Complex<double> type}
\uncover<4->{Complex<float> f;              // different type than d}
\uncover<5->{Complex e\{1.,1.\};              // Complex<double> deduced}
\uncover<6->{d + e;                         // ok}
\uncover<7->{d + f;                         // possibly error}
\uncover<8->{Complex<int> i;                // \alt<8>{acceptable?}{\alert<9>{error}}}}\end{codeblock}

\uncover<9>{\scriptsize\begin{itemize}
  \item[*] compile-time check + type introspection
  \end{itemize}}

\end{frame}

\begin{frame}{Class template \insertcontinuationtext}
  \begin{itemize}
  \item A class template can be seen as a type \textit{generator}
  \item A class template is parameterized by one or more template parameters
    (types, but not only)
  \item A template \textit{specialization} identifies a template name and a set
    of template arguments corresponding to the template parameters
    \begin{itemize}
    \item Template arguments can be explictly specified (e.g. following the
      template name between angular brackets \code{<>}) or deduced based on
      constructor arguments
    \end{itemize}
  \item When a class template specialization is used in a context where a type
    is needed, the compiler \textit{instantiates} a new type
    \begin{itemize}
    \item Identical instantiations are merged together
    \end{itemize}
  \item Only member functions that are actually used are instantiated
  \item There are ways to constrain the arguments used to instantiate a template
    \begin{itemize}
    \item Notably, \Cpp{}20 has introduced \textit{concepts}
    \end{itemize}
  \end{itemize}
\end{frame}

\begin{frame}[fragile]{Function template}

  \begin{codeblock}<1->{
double norm2(Complex const\& c) \{ \ddd \} // not valid any more
norm2(f);                             // error; f is of type Complex<float>}\end{codeblock}

  \begin{codeblock}<2->{
auto norm2(Complex<\alert<4-5>{float}> const\& c) \{
  return c.real() * c.real() + c.imag() * c.imag();
\}
norm2(f); // ok
norm2(d); // error; d is of type Complex<double>}\end{codeblock}

  \begin{codeblock}<3->{
auto norm2(Complex<\alert<4-5>{double}> const\& c) \{
  return c.real() * c.real() + c.imag() * c.imag();
\}
norm2(d); // ok}\end{codeblock}

  \begin{codeblock}<5->{
template<typename \alert<5>{FP}> // or template<class FP>
auto norm2(Complex<\alert<5>{FP}> const\& c) \{
  return c.real() * c.real() + c.imag() * c.imag();
\}
\uncover<6>{auto nf = norm2(f); // nf is of type float
auto nd = norm2(d); // nd is of type double}}\end{codeblock}

\end{frame}

\begin{frame}[fragile]{Function template \insertcontinuationtext}

  \begin{codeblock}
template<typename \alert<2>{FP}>
auto norm2(\alert<2>{Complex<FP>} const\& c)
\{ return c.real() * c.real() + c.imag() * c.imag(); \}

\uncover<2->{template<typename \alert<2>{C}>
auto norm2(\alert<2>{C} const\& c)
\{ return c.real() * c.real() + c.imag() * c.imag(); \}}

\uncover<3->{norm2(f); // ok}
\uncover<4->{norm2(d); // ok}
\uncover<5->{std::complex<float> g;
norm2(g); // ok!}

\uncover<6->{namespace std \{
  template<class T>
  class complex \{
    \ddd
   public:
    T real() const;
    T imag() const;
  \};
\}}\end{codeblock}

  \begin{textblock}{6}(6,5)
    \visible<7->{This shows the basic idea behind \alert<7>{generic programming}}
  \end{textblock}

\end{frame}

\begin{frame}{Function template \insertcontinuationtext}
  \begin{itemize}
  \item A function template can be seen as a function \textit{generator}
  \item A function template is parameterized by one or more template parameters
    (types, but not only)
  \item A template \textit{specialization} identifies a template name and a set
    of template arguments corresponding to the template parameters
    \begin{itemize}
    \item Template arguments can be explictly specified (e.g. following the
      template name between angular brackets \code{<>}) or deduced based on
      function arguments
    \end{itemize}
  \item When a function template specialization is used in a context where a
    function is needed, the compiler \textit{instantiates} a new function
    \begin{itemize}
    \item Identical instantiations are merged together
    \end{itemize}
  \item Only functions that are actually used are instantiated
  \item There are ways to constrain the arguments used to instantiate a template
  \item Usually there is no separate definition of a function template
  \end{itemize}
\end{frame}

\begin{frame}[fragile]{Template argument deduction}

  \begin{itemize}
  \item In order to instantiate a template every template argument has to be
    known
  \item Template arguments can be deduced from function call arguments
    \begin{codeblock}
template<class F> F norm2(Complex<F> const& c) \{\ddd\}

Complex<float> f;
norm2(f);          // no need to specify norm2<float>(f)
norm2<float>(f);   // ok, be explicit
norm2<double>(f);  // instantiate and call norm2<double>, probably failing\end{codeblock}

  \item Sometimes it's not possible to deduce all template arguments from function
    parameters
    \begin{codeblock}
template<class Number>
Number lexical_cast(std::string const& s) // convert a string to a number
\{\ddd\}

auto d = lexical_cast<double>("3.14"); // the destination type is needed\end{codeblock}
  \end{itemize}
\end{frame}

\begin{frame}[fragile]{Template argument deduction \insertcontinuationtext}

  \begin{itemize}
  \item Constructor calls can be used to deduce \textit{class} template arguments
    \begin{itemize}
    \item Constructor Template Argument Deduction (CTAD)
    \end{itemize}
    \begin{codeblock}
template<class FP> class Complex \{\ddd\};

Complex d;             // error, cannot deduce FP
Complex<double> e;     // ok
Complex f\{1.\};         // ok, Complex<double>
Complex<double> g\{1.\}; // ok
Complex<float> h\{1.\};  // ok, 1. -> 1.F\end{codeblock}
  \item If CTAD is not available (e.g. in older \Cpp{} versions), leverage
    function template argument deduction
    \begin{codeblock}
template<class FP>
auto make_complex(FP r, FP i) \{
  return Complex<FP>(r, i);
\}

auto c = make_complex(1.,2.); // instead of Complex c\{1.,2.\}\end{codeblock}
  \end{itemize}

\end{frame}

\begin{frame}{Exercises}

  \begin{itemize}
  \item Write an \code{operator+} that adds two complex numbers of different types

  \item Transform the \code{Rational} class into a class template, where the
    parameter is the type used to represent numerator and denominator. Adapt the
    operators.

    Write a program that instantiates the \code{Rational} class with integral
    types such as \code{int}, \code{long int}, \code{long long int}, \code{short
      int}. Combine objects of those types using the operators.
  \end{itemize}

\end{frame}

\begin{frame}[fragile]{Non-type template parameters}
  \begin{itemize}
  \item A class/function template parameter need not be a type
  \item A template parameter can be a value, which must be known at compile time

    \begin{codeblock}<2->{
template<int I> struct C \{ \ddd \};
C<2> c2;
C<3> c3;              // different type
int n; std::cin >> n;
C<n> c4;              // error, n is not a compile-time constant}\end{codeblock}

  \item<3-> The type of the value needs to satisfy certain constraints
  \item<3-> Integral types (e.g. \code{int}, \code{char}, \code{bool}) and
    enumeration types are allowed, other \textit{structural} types (including
    floating-point types) since \Cpp{}20

    \begin{codeblock}<3->{
template<double D> struct F \{ \ddd \}; // ok since C++20}\end{codeblock}

\item<4-> Type and non-type template parameters can be mixed

  \begin{codeblock}<4->{
template<class T, int N> struct array \{ \ddd \};
template<class T, T v> struct integral_constant \{ \ddd \};}\end{codeblock}

  \end{itemize}

\end{frame}
