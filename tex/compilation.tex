% !TeX root = ./pf2022.tex

%\includeonlyframes{current}

\section{Compilation model}

\begin{frame}[fragile]{The C++ compilation model}

  This is the \textbf{current} model; \Cpp{}20 has introduced \textit{modules}
  (not further discussed)

  \begin{center}
    \begin{tikzpicture}[every text node part/.style={align=center},
      source/.style={
        tape,tape bend height=.15cm,draw,thick,inner sep=4pt
      }
      ]

      \node[source] (source) {Source file \\
        {\scriptsize(\code{hello.cpp})}};

      \node[source,right=3cm of source] (exe) {Executable\\binary file \\ {\scriptsize(\code{a.out})}};

      \draw[thick,-Stealth] (source) -- (exe) node[above,align=center,midway] {Compilation \\ {\scriptsize(\code{g++})}};


      \visible<2->{\node[source,left=2cm of source,yshift=1cm] (iostream) {Header \\
          {\scriptsize(\code{iostream})}\\\hphantom{\scriptsize(\code{string})}};

        \node[source,left=2cm of source,yshift=-1cm] (string) {Header \\
          {\scriptsize(\code{string})}\\\hphantom{{\scriptsize(\code{iostream})}}};

        \draw[thick,dashed,-{Stealth[open]}] (string) -- (source) node[below,midway] {\tiny\code{\#include}};
        \draw[thick,dashed,-{Stealth[open]}] (iostream) -- (source) node[above,midway] {\tiny\code{\#include}};
      }
      \visible<3->{
        \node[inner sep=10pt,rounded corners,draw=red,thick,fit=(iostream) (string) (source)] (tu) {};
        \node[red,above left=2mm of tu.south east,anchor=south east] {Translation Unit};
      }
    \end{tikzpicture}
  \end{center}

  \begin{itemize}
  \item<4-> The production of the \textit{translation unit} is the first step of
    the overall \textit{compilation} process
  \item<4-> It is accomplished by running the \textit{preprocessor} on a source file
  \item<5-> To stop the compilation process after the preprocessing:
    \begin{itemize}
    \item \code{g++ -E hello.cpp}
    \end{itemize}
  \end{itemize}

\end{frame}

\begin{frame}[fragile]{The C++ compilation model \insertcontinuationtext}

  \begin{itemize}
  \item A typical C++ program is composed of many files, i.e. there are many
    translation units
    \begin{center}
      \visible<2->{\begin{tikzpicture}[
          every text node part/.style={align=center},
          every node/.style={fill=white,inner sep=3pt},
          source/.style={
            tape,tape bend height=.15cm,draw=red,thick
          },
          binary/.style={
            tape,tape bend height=.15cm,draw=green,thick
          },
          tool/.style={
            single arrow, single arrow head extend=0, draw,thick,
            shape border rotate=0,single arrow tip angle=120
          }
          ]

          \node[source] (sources) {\phantom{Translation}\\\phantom{Units}};
          \foreach \x in {4,...,1} {
            \node[source,very thin,above left=\x mm and \x mm of sources,
            anchor=north west] {\phantom{Translation}\\\phantom{Units}};
          }
          \node[source] at (sources) {Translation\\Units};

          \node[tool,right=1cm of sources] (compiler) {\textit{Compiler}};

          \node[binary,right=1cm of compiler] (exe) {Executable};

          \draw[thick] (sources) -- node[near start,above,inner sep=0,outer sep=1pt] {\scriptsize\textbf{+}} (compiler);
          \draw[thick] (compiler) -- (exe);
        \end{tikzpicture}
      }
    \end{center}
  \item<3-> How to distribute source code over multiple files?
  \item<3-> How to expose functionality from one translation unit so that it can
    be used in another?
  \item<4-> \textbf{Physical} design
  \end{itemize}

\end{frame}

\begin{frame}[fragile]{The C++ compilation model \insertcontinuationtext}

  \begin{itemize}
  \item The overall process to translate C++ source code to the final binary
    executable consists of multiple stages
  \end{itemize}

  \begin{center}
      \begin{tikzpicture}[
        every text node part/.style={align=center},
        every node/.style={fill=white,inner sep=3pt},
        source/.style={
          tape,tape bend height=.15cm,draw=red,thick
        },
        binary/.style={
          tape,tape bend height=.15cm,draw=green,thick
        },
        object/.style={
          tape,tape bend height=.15cm,draw=green!50!black,thick
        },
        tool/.style={
          single arrow, single arrow head extend=0,draw,thick,
          shape border rotate=0,single arrow tip angle=120
        }
        ]

        \node[source] (sources) {\phantom{Translation}\\\phantom{Units}};
        \foreach \x in {4,...,1} {
          \node[source,very thin,above left=\x mm and \x mm of
          sources,anchor=north west]
          {\phantom{Translation}\\\phantom{Units}};
        }
        \node[source] at (sources) {Translation\\Units};


        \node[tool,right=.5cm of sources] (compiler) {Compiler};

        \node[object,right=1cm of compiler,draw=none] (objects) {\phantom{Object}\\\phantom{Files}};
        \foreach \x in {4,...,1} {
          \node[object,very thin,above left=\x mm and \x mm of
          objects,anchor=north west] {\phantom{Object}\\\phantom{Files}};
        }
        \node[object] at (objects){Object\\Files};

        \node[tool,right=.5cm of objects] (linker) {Linker};

        \node[binary,right=.5cm of linker] (exe) {Executable};

        \draw[thick] (sources) -- node[near start,above,inner sep=0,outer sep=1pt] {\scriptsize\textbf{+}} (compiler);
        \draw[thick] (compiler) -- node[near start,above,inner sep=0,outer sep=1pt] {\scriptsize\textbf{+}} (objects);
        \draw[thick] (objects) -- node[near start,above,inner sep=0,outer sep=1pt] {\scriptsize\textbf{+}} (linker);
        \draw[thick] (linker) -- (exe);

        \node[object,above left=1.3cm and 0cm of linker] (libstd) {Standard\\Library};

        \node[object,right=.6cm of libstd,draw=none] (otherlibsph) {\phantom{Other}\\\phantom{Libraries}};
        \foreach \x in {4,...,1} {
          \node[object,very thin,above left=\x mm and \x mm of
          otherlibsph,anchor=north west]
          {\phantom{Other}\\\phantom{Libraries}};
        }
        \node[object] at (otherlibsph) {Other\\Libraries};

        \draw[thick,-Stealth] (libstd) -- (linker);
        \draw[thick,-Stealth] (otherlibsph) -- (linker);

      \end{tikzpicture}
    \end{center}

    \begin{itemize}
    \item<2-> To produce object files without linking:
      \begin{itemize}
      \item \code{g++ -c hello.cpp}
      \end{itemize}
    \item<2-> To keep all the intermediate files generated during the compilation process:
      \begin{itemize}
      \item \code{g++ -save-temps hello.cpp}
      \end{itemize}
    \end{itemize}

\end{frame}

\begin{frame}[fragile]{Definition vs Declaration}

  \begin{itemize}
  \item A \alert{definition} is a declaration that fully defines an entity
  \item Examples:
  \end{itemize}

  \begin{codeblock}{
bool less_than(int a, int b) \{ // function definition
  return a < b;
\}

class Sample \{ // class definition
 public:
  void add(double x) // member function definition
  \{
    ++N_;
    sum_x_ += x;
    \ddd
  \}
  \ddd
\};
}\end{codeblock}

\end{frame}

\begin{frame}[fragile]{Definition vs Declaration \insertcontinuationtext}

  \begin{itemize}
  \item A \alert{declaration} that is \textbf{not} a definition just introduces
    the name (and, when applicable, the type) of the entity
  \item Examples:
  \end{itemize}

  \begin{codeblock}{
bool less_than(int a, int b); // function declaration (prototype)

class Sample; // class (forward) declaration

class Sample \{ // class definition
 public:
  void add(double x); // member function declaration
  \ddd
\};
}\end{codeblock}

\end{frame}

\begin{frame}{One-Definition Rule (ODR)}
  \begin{itemize}[<+->]
  \item An entity can be defined only once in each translation unit
  \item More generally, an entity can be defined only once in the whole program,
    with exceptions
  \item Some entities can be defined in multiple translation units, provided the
    definitions are identical (\textit{token-by-token} in the source code)
    \begin{itemize}[<.->]
    \item Class definitions
    \item \textit{inline} functions and variables
    \item Class and function templates
    \item \ldots
    \end{itemize}
  \item To guarantee that definitions are identical in all translation units
    they typically appear in \textbf{header files}, which are then
    \code{\#include}d where needed
  \item The compiler may not be able to diagnose violations of the ODR
    \begin{itemize}
    \item The generated executable may behave incorrectly
    \end{itemize}
  \end{itemize}
\end{frame}

\begin{frame}{Definition vs Declaration during compilation and linking}

  \begin{itemize}[<+->]
  \item During the compilation step:
  \begin{itemize}
  \item Calling a function requires the previous declaration of that function
  \item The declaration of a function requires the previous declaration of the
    types involved
  \item Creating and manipulating an object require the definition of the
    corresponding type (the type has to be \textit{complete})
  \end{itemize}

\item During the linking step:

  \begin{itemize}
  \item Everything has to be properly defined
  \end{itemize}
    \end{itemize}

\end{frame}

\begin{frame}{Header and source files}

  \begin{itemize}[<+->]
  \item Header files are the primary mechanism to guarantee that declarations
    and definitions are identical in all translation units
    \begin{itemize}[<.->]
    \item Because they are \code{\#include}d in other header and source files
      where those declarations and definitions are needed
    \end{itemize}
  \item For a given software component, a typical situation foresees
    \begin{itemize}
    \item One \textbf{header} file containing free function declarations, class
      definitions with member function declarations, template definitions
    \item One \textbf{source} file containing definitions of free and member functions
      and of any other entity needed for the implementation
      \begin{itemize}
      \item \textbf{Suggestion} the source file \code{\#include}s the header
        file as the first \code{\#include}
      \end{itemize}
    \item One file with the unit \textbf{tests}
    \end{itemize}
  \item A function defined in a header file needs to be
    \code{inline}
    \begin{itemize}
    \item NB the \code{inline} keyword is not (any more) an optimization hint to
      the compiler
    \end{itemize}
  \item Member functions (methods) defined inside the class and function
    templates are implicitly \code{inline}
  \end{itemize}

\end{frame}

\begin{frame}[fragile]{Member functions defined outside the class}

  \begin{itemize}

  \item A method must be declared inside the class, but can be \textbf{just}
    declared inside the class, i.e. not also defined

    \begin{codeblock}
class Sample \{
  \ddd
  void add(double);
\};\end{codeblock}

  \item When defining a method outside the class, the method name must be
    prefixed with the class name followed by the \textit{scope} operator
    \code{::}, i.e. \textit{class-name}\code{::}\textit{method-name}

    \begin{codeblock}
void Sample::add(double x) \{
  ++N_;
  sum_x_ += x;
  \ddd
\}\end{codeblock}

  \end{itemize}

\end{frame}

\begin{frame}[fragile]{Member functions defined outside the class \insertcontinuationtext}

  \begin{itemize}
  \item The method can be defined in the same file (normally a header file) or
    even in another file (normally the corresponding source file)

  \item If the definition is in a header file, the method should be declared \code{inline}

    \begin{codeblock}
\alert{inline} void Sample::add(double d) \{
  ++N_;
  sum_x_ += x;
  \ddd
\}\end{codeblock}

    This informs the compiler/linker that all the definitions of the function, even
    if they are in multiple translation units, are equal

  \end{itemize}

\end{frame}

\begin{frame}[fragile]{Include guards}

  \begin{itemize}[<+->]
  \item A header (file) may be included, directly or indirectly, multiple times
    in the \textbf{same} translation unit
    \begin{codeblock}
#include "result.hpp"
#include "sample.hpp" // sample.hpp already includes result.hpp\end{codeblock}

  \item Multiple definitions of the same entity would be available in the
    \textbf{same} translation unit, which is illegal
  \item Placing an \textit{include guard} at the beginning of each header file
    prevents multiple inclusions in the \textbf{same} translation unit
    \begin{codeblock}
// file result.hpp
#ifndef RESULT_HPP
#define RESULT_HPP
\ddd
// classes, functions, templates, \ldots
\ddd
#endif\end{codeblock}
  \end{itemize}
\end{frame}

\begin{frame}{To inline or not to inline?}

  \begin{itemize}
  \item Pros:
    \begin{itemize}
    \item A compiler can optimize more because it sees more code $\rightarrow$
      faster execution
      \begin{itemize}
      \item Definitions of inline functions contained in included header files
        become part of the translation unit
      \end{itemize}
    \item Header-only libraries (i.e. everything is in one or more header files)
      are easier to distribute (e.g. \code{doctest})
    \end{itemize}
  \item Cons:
    \begin{itemize}
    \item More physical coupling between software components
      \begin{itemize}
      \item Visibility of implementation details
      \item Every time a header file changes all the including files need to be
        recompiled $\rightarrow$ longer compilation times
      \end{itemize}
    \item The code of the inlined functions may be included many times in the
      final binary, making it larger and causing inefficient use of the memory
      system $\rightarrow$ slower execution
    \end{itemize}
  \end{itemize}

\end{frame}

\begin{frame}{Build systems}

  \begin{itemize}
  \item When a project includes more than a few files, keeping track of
    compilation dependencies between them becomes difficult
  \item A project may foresee multiple \textit{artifacts}: executables,
    libraries (i.e. collection of object files), tests, \ldots
  \item A project may depend on other projects, e.g. other libraries
  \item Usually multiple variants of a project are needed: one for debug, one
    for testing with code coverage, one for release, one for measuring
    performance (\textit{profiling}), \ldots
    \begin{itemize}
    \item Each variant foresees different compilation flags
    \end{itemize}
  \item \ldots
  \end{itemize}

  \begin{itemize}
  \item In practice, all the above cannot be done manually and an automatic
    system is required
  \item Such a system is usually called \textbf{build system}
  \item Many build systems exist. We'll use
    \href{https://cmake.org}{\textbf{CMake}}
  \end{itemize}

\end{frame}

\begin{frame}[fragile]{CMake}
  \begin{itemize}
  \item \textit{CMake is an open-source, cross-platform family of tools designed
      to build, test and package software}
  \item Most used tool to build \Cpp{} projects
  \item Build targets, dependencies, options, \ldots are expressed declaratively
    in a file called \code{CMakeLists.txt}
    \begin{codeblock}{
cmake_minimum_required(VERSION 3.16)
project(mandelbrot_sfml VERSION 0.1.0)

find_package(SFML 2.5 COMPONENTS graphics REQUIRED)

string(APPEND CMAKE_CXX_FLAGS " -Wall -Wextra")

add_executable(mandelbrot_sfml main.cpp)
target_link_libraries(mandelbrot_sfml PRIVATE sfml-graphics)
}\end{codeblock}

  \item See
    \href{https://github.com/Programmazione-per-la-Fisica/pf2022/tree/main/code/mandelbrot_sfml}{code/mandelbrot_sfml}
    for a basic example
  \end{itemize}
\end{frame}