% !TeX root = ./pf2022.tex

%\includeonlyframes{current}

\section{Pointers and references}

\begin{frame}[fragile]{Passing by value may be inconvenient}

  \begin{itemize}
  \item Consider a function that increments an \code{int} object

  \begin{codeblock}
\uncover<2->{void increment(\alert{???} n) \{}
\uncover<2->{  // ++n}
\uncover<2->{\}}

int number\{42\};
increment(number);
// number == 43\end{codeblock}

  \item NB The function does not \textbf{return} the new value; it modifies the
    passed object \textbf{in place}

  \item<2-> We cannot write \code{void increment(int n)}, because the function would
    modify a \textbf{copy} of the original object
  \end{itemize}

\end{frame}

\begin{frame}[fragile]{Passing by value may be expensive}

  Let's consider a function that counts the number of words in a string

  \begin{columns}
    \begin{column}{.5\textwidth}

      \begin{codeblock}
int count_words(std::string \alert<5>{s})
\{
  int count\{0\};
  \ddd s \ddd
  return count;
\}

\alert<2|trans:0>{int main()
\{
  std::string text}\alert<3|trans:0>{\{\ddd\}};
  \alert<2|trans:0>{int const res}\{\alert<4|trans:0>{count_words(}\alert<5>{text}\alert<4|trans:0>{)}\};
  std::cout <{}< res <{}< \bslashn;
\alert<2|trans:0>{\}}\end{codeblock}

      \begin{itemize}
      \item<5-> \code{s} is a \textbf{copy} of \code{text}
      \item<6-> I'm slightly cheating: not all string memory goes on the stack
      \end{itemize}

    \end{column}

    \begin{column}{.4\textwidth}

      \begin{tikzpicture}[
        mem/.style={
          minimum width=2cm,
          inner sep=0pt,
          outer sep=0pt,
          draw=black
        },
        frame/.style={
          minimum width=2cm,
          inner sep=0pt,
          outer sep=0pt,
          draw=black,
          thick,
          fill=green!50!white
        },
        var/.style={
          node font=\ttfamily\scriptsize,
          minimum height=.5cm,
          minimum width=2cm,
          draw=black,
          fill=green!30!white,
          inner sep=0pt,
          outer sep=0pt
        },
        anchor=south west
        ]
        \node at (0,0) [
          mem,
          minimum height=6cm
        ] {};
        \visible<2->{
          \node (main) at (0,3) [
            frame,
            minimum height=2.5cm,
            label={[yshift=1.1cm]0:\scriptsize\tt main}
          ] {};
          \node (result) [
            var,
            above=3ex of main.south,
            label={180:\scriptsize\tt res}
          ] {};
          \node (text) [
            var,
            minimum height=1cm,
            text width=2cm,
            align=center,
            above=0pt of result,
            label={180:\scriptsize\tt text}
          ] {\only<3->{\tiny\textit{Nel mezzo del cammin di nostra vita\ldots}}};
        }
        \visible<4->{
          \node (countwords) [
            frame,
            below=0pt of main,
            minimum height=2.5cm,
            label={[yshift=1.1cm]0:\scriptsize\tt count_words}
          ] {};
          \node (count) [
            var,
            above=3ex of countwords.south,
            label={180:\scriptsize\tt count}
          ] {};
          \node (s) [
            var,
            minimum height=1cm,
            text width=2cm,
            align=center,
            above=0pt of count,
            label={180:\scriptsize\tt s}
          ] {\only<5->{\tiny\textit{Nel mezzo del cammin di nostra vita\ldots}}};
        }
      \end{tikzpicture}
    \end{column}

  \end{columns}
\end{frame}

\begin{frame}{Hexadecimal notation}

  \begin{columns}
    \begin{column}{.75\textwidth}
      \begin{itemize}
      \item<1-> Numbers in hexadecimal notation are represented in base sixteen
      \item<2-> Each digit has a value between zero and fifteen (included)
      \item<2-> Sixteen symbols are needed for the digits
        \begin{itemize}
        \item 0 ... 9, A, B, C, D, E, F (also lowercase)
        \end{itemize}
      \item<3-> In C++ numeric literals (i.e. constants) can be expressed in hexadecimal
        notation, prepending \code{0x} (or \code{0X}) to the hexadecimal representation
        \begin{itemize}
        \item \code{0x9 == 9}, \code{0xA == 10}, \code{0x10 == 16}
        \end{itemize}
      \item<3-> Easy conversion between hex and binary representation: each hex digit
        corresponds to four bits
      \item<4-> Hex literals are typically used to represent raw contents of memory or
        memory addresses
      \end{itemize}

    \end{column}

    \begin{column}{.25\textwidth}
      \visible<3->{
        \small{
          \begin{tabular}[t]{|r|r|r|}
            Hex & Dec & Bin \\
            \hline
            0 &  0 & 0000 \\
            1 &  1 & 0001 \\
            2 &  2 & 0010 \\
            3 &  3 & 0011 \\
            4 &  4 & 0100 \\
            5 &  5 & 0101 \\
            6 &  6 & 0110 \\
            7 &  7 & 0111 \\
            8 &  8 & 1000 \\
            9 &  9 & 1001 \\
            A & 10 & 1010 \\
            B & 11 & 1011 \\
            C & 12 & 1100 \\
            D & 13 & 1101 \\
            E & 14 & 1110 \\
            F & 15 & 1111
          \end{tabular}
        }
      }
    \end{column}
  \end{columns}

\end{frame}

\begin{frame}[fragile,label=current]{Pointers}

  Where in memory does a given object reside?

  \begin{tikzpicture}[
      mem/.style={
        node font=\ttfamily\scriptsize,
        minimum height=.5cm,
      },
      location/.style={
        mem,
        draw=black!50,
        minimum width=1.2cm,
        fill=green!20!white,
      },
      address/.style={
        mem,
        draw=black,
        minimum width=.3cm,
        densely dotted,
      },
      every node/.style={
        mem,
      },
      anchor=south west,
      node distance=0,
    ]

    \visible<2->{\node (memory) at (0,0) [mem, draw=black!50, minimum width=.9\textwidth] {};}
    \visible<3->{\node (begin mem) [address, right=of memory.west, "0x0000" above] {};}
    \visible<3->{\node (end mem) [address, left=of memory.east, "0xffff" above] {};}

    \visible<6->{\node (pp) at (1.2,0) [location, "pp" below] {\alert<6|trans:0>{0xbad0}};}
    \visible<6->{\node (addr pp) [address, right=of pp.west, "0xaa04" above] {};}

    \visible<2->{\node (i) [location, right=0.5 of pp, "i" below,] {4321};}
    
    % the colon between braces below is needed to avoid that the label command
    % interprets the colon
    % search "Handling commas and colons inside the text" in the tikz manual
    \visible<3->{\node (addr i) [address, right=of i.west,
      "\alert<5|trans{:}0>{0xab00}" above] {};}

    \visible<5->{\node (p) [location, right=0.5 of i, "p" below] {
        \alt<-6>{\alert<5|trans:0>{0xab00}}{\alert<7-8|trans:0>{0xcd00}}};
    }
    \visible<5->{\node (addr p) [address, right=of p.west, "\alert<6|trans{:}0>{0xbad0}" above] {};}

    \visible<2->{\node (j) [location, right=0.5 of p, "j" below,] {1234};}
    \visible<3->{\node (addr j) [address, right=of j.west, "\alert<7|trans{:}0>{0xcd00}" above] {};}

    \visible<8->{\node (q) [location, right=0.5 of j, "q" below,] {\alert<8|trans:0>{0xcd00}};}
    \visible<8->{\node (addr q) [address, right=of q.west, "0xcb80" above] {};}

    \draw<5-6|trans:0>[{Circle[length=2pt]}-Stealth,opacity=.6] ([yshift=.1cm]p.center) .. controls +(-1,.5) .. (addr i.north);
    \draw<7->[{Circle[length=2pt]}-Stealth,opacity=.6] ([yshift=.1cm]p.center) .. controls +(1,.5) .. (addr j.north);
    \draw<6->[{Circle[length=2pt]}-Stealth,opacity=.6] ([yshift=-.1cm]pp.center) .. controls +(2,-.5) .. (addr p.south);
    \draw<8->[{Circle[length=2pt]}-Stealth,opacity=.6] ([yshift=.1cm]q.center) .. controls +(-1,.5) .. (addr j.north);

  \end{tikzpicture}

  \begin{codeblock}<2->{
\visible<2->{int i\{4321\};}
\visible<2->{int j\{1234\};}
\visible<4->{std::cout <{}< \alert<4>{&}i;  // 0xab00, \alert<4>{address-of operator}}
\visible<4->{std::cout <{}< \alert<4>{&}j;  // 0xcd00}
\visible<5->{int\alert<5>{*} p\{&i\};       // \alert<5>{pointer declarator}}
\visible<6->{std::cout <{}< &p;  // 0xbad0}
\visible<6->{int\alert<5>{**} pp\{&p\};     // &p is of type int**}
\visible<7->{p = &j;}
\visible<8->{int* q\{p\};        // p and q point to the same object}}\end{codeblock}

  \visible<5->{Note that \code{int*} is a new type}\visible<6->{ and so is
    \code{int**}}


\end{frame}

\begin{frame}[fragile,label=current]{Pointers \insertcontinuationtext}
  
  \begin{tikzpicture}[
      mem/.style={
        node font=\ttfamily\scriptsize,
        minimum height=.5cm,
      },
      location/.style={
        mem,
        draw=black!50,
        minimum width=1.2cm,
        fill=green!20!white,
      },
      address/.style={
        mem,
        draw=black,
        minimum width=.3cm,
        densely dotted,
      },
      every node/.style={
        mem,
      },
      anchor=south west,
      node distance=0,
    ]

    \visible<1->{\node (memory) at (0,0) [mem, draw=black!50, minimum width=.9\textwidth] {};}
    \visible<1->{\node (begin mem) [address, right=of memory.west, "0x0000" above] {};}
    \visible<1->{\node (end mem) [address, left=of memory.east, "0xffff" above] {};}

    \visible<3->{\node (k) at (1.2,0) [location, "k" below,] {\alert<3|trans:0>{1234}};}
    \visible<3->{\node (addr k) [address, right=of k.west, "0xaacc" above] {};}

    \visible<1->{\node (p) [location, right=0.5 of k, "p" below] {0xcd00};}
    \visible<1->{\node (addr p) [address, right=of p.west, "0xbad0" above] {};}

    \visible<1->{\node (j) [location, right=0.5 of p, "j" below,] {
        \alt<-3>
            {1234}
            {\alt<4>
              {\alert<4|trans:0>{5678}}
              {\alert<5|trans:0>{5679}}}};
    }
    \visible<1->{\node (addr j) [address, right=of j.west, "0xcd00" above] {};}

    \visible<6-|trans:0>{\node (q) [location, right=0.5 of j, "q" below,] {\alt<6>{\alert<6>{0xcd00}}{\alert<7>{\textit{null}}}};}
    \visible<trans>{\node (q) [location, right=0.5 of j, "q" below,] {\textit{null}};}
    \visible<6->{\node (addr q) [address, right=of q.west, "0xcb80" above] {};}

    \draw<1->[{Circle[length=2pt]}-Stealth,opacity=.6] ([yshift=.1cm]p.center) .. controls +(1,.5) .. (addr j.north);
    \draw<6|trans:0>[{Circle[length=2pt]}-Stealth,opacity=.6] ([yshift=.1cm]q.center) .. controls +(-1,.5) .. (addr j.north);

  \end{tikzpicture}

  \begin{codeblock}
\visible<1->{int j\{1234\};}
\visible<1->{int* p\{&j\};}
\visible<2->{std::cout <{}< \alert<2->{*}p; // 1234, \alert<2>{dereference} operator}
\visible<3->{int k\{\alert<3->{*}p\};       // k is a copy of j}
\visible<4->{\alert<4->{*}p = 5678;}
\visible<5->{++\alert<5->{*}p;            // or ++(*p) for more clarity}
\visible<6->{int* q\{p\};}
\visible<7->{q = nullptr;}
\visible<8->{\alert<8->{*}q;              // \alert{undefined behavior}}\end{codeblock}

  \visible<8->{Dereferencing a null pointer is a logical error (i.e. a bug)}

\end{frame}

\begin{frame}[fragile]{Pointers \insertcontinuationtext}

  \begin{itemize}
  \item<1-> \textit{address-of} operator: \alert{\code{\&}}
    \begin{itemize}
    \item Given an object, it returns its address in memory
    \end{itemize}
  \item<2-> \textit{dereference} operator: \alert{\code{*}}
    \begin{itemize}
    \item Given a pointer to an object, it gives access to that very object
    \end{itemize}
  \end{itemize}
\end{frame}

\begin{frame}[fragile]{Passing a pointer}

  \begin{columns}[T]
    \begin{column}{.5\textwidth}
      \begin{codeblock}
void increment(int\alert{*} n) \{
  ++(\alert{*}n);
\}

int number\{42\};
increment(\alert{&}number);
// number == 43\end{codeblock}
      
    \end{column}
    \begin{column}{.5\textwidth}
      \begin{codeblock}
int count_words(std::string\alert{*} s)
\{
  int count\{0\};
  \ddd \alert{*}s \ddd
  return count;
\}

std::string text\{\ddd\};
count_words(\alert{&}text);\end{codeblock}
    \end{column}
  \end{columns}

  \begin{itemize}
  \item The caller takes the address of the object and passes it to the function
  \item The function dereferences the pointer to get access to the object
  \item Be careful not to dereference a null pointer
  \end{itemize}

\end{frame}

\begin{frame}[fragile]{References}

  \begin{itemize}
  \item A variable declared as a \alert{reference} of a type \code{T} is another
    name (an \textit{alias}) for an existing object of type \code{T}
  \end{itemize}

  \begin{tikzpicture}[
      mem/.style={
        node font=\ttfamily\scriptsize,
        minimum height=.5cm,
      },
      location/.style={
        mem,
        draw=black!50,
        minimum width=2cm,
        fill=green!20!white,
      },
      every node/.style={
        mem,
      },
      anchor=south west,
      node distance=0,
    ]

    \visible<2->{
      \node (memory) at (0,0) [mem, draw=black!50, minimum width=\textwidth] {};

      \node (i) at (2,0) [
        location,
        "i" below,
        "\alert<3>{ri}" {visible on=<3->,yshift=-0.3cm,below}
      ] {\only<-4|trans:0>{12}\only<5-6|trans:0>{34}\only<7-|trans>{56}};

      \node (j) at (7,0) [
        location,
        "j" below
      ] {56};
    }

  \end{tikzpicture}

  \begin{codeblock}<2->{
int i\{12\};
int j\{56\};
\uncover<3->{int\alert<3>{\&} ri\{i\};          // reference declarator}
\uncover<4->{ri == 12;            // true}
\uncover<5->{ri = 34;}
\uncover<6->{ri == 34 \&\& i == 34; // true}
\uncover<7->{ri = j;}
\uncover<8->{ri == 56 \&\& i == 56; // true}
\uncover<9->{int& r;              // error}}\end{codeblock}

\end{frame}

\begin{frame}[fragile]{References \insertcontinuationtext}

  \begin{itemize}[<+->]
  \item A reference must be initialized to refer to a valid object
    \begin{itemize}[<.->]
    \item A reference cannot be \textit{null}
    \end{itemize}
  \item A reference cannot \textit{rebind} (be re-associated) to another object
  \item For a given type \code{T}, \code{T\&} is a \textit{compound} type,
    distinct from \code{T}
  \item Dereferencing a pointer gives back a reference to that object
    \begin{codeblock}<.->{
int i\{42\};
int* p\{&i\};
int& r\{*p\}; // i and r are names for the same object}\end{codeblock}
  \end{itemize}

\end{frame}

\begin{frame}[fragile]{Passing by reference}

  \begin{codeblock}
void increment(int\alert{&} n) \{
  ++\alert{n};
\}

int number\{42\};
increment(\alert{number});
// number == 43\end{codeblock}

  \begin{itemize}
  \item There is no difference in the caller compared to pass-by-value
  \item There is no difference in the body of the function compared to pass-by-value
  \item The only visual clue is in the parameter declaration
  \end{itemize}

\end{frame}

\begin{frame}[fragile]{Passing by reference \insertcontinuationtext}
  \begin{columns}
    \begin{column}{.5\textwidth}

      \begin{codeblock}
int count_words(std::string\alert<1->{\&} s)
\{
  \alert<5|trans:0>{int count\{0\};}
  \alert<6|trans:0>{\ddd} \alert<6>{s} \alert<6|trans:0>{\ddd} // just use s
  \alert<7|trans:0>{return count;}
\}

\alert<2|trans:0>{int main()
\{
  std::string text}\alert<3|trans:0>{\{\ddd\}};
  \alert<2|trans:0>{int const res}\{\alert<4|trans:0>{count_words(}\alert<4>{text}\alert<4|trans:0>{)}\};
  std::cout <{}< res <{}< \bslashn;
\alert<2|trans:0>{\}}\end{codeblock}

      \uncover<8>{}
      \uncover<9>{}

      \vskip .5 cm

      \visible<4->{\scriptsize (*) In general it's unspecified if a reference
        occupies storage. Here probably it does and a pointer (called s$'$) is
        passed behind the scenes.}

    \end{column}

    \begin{column}{.4\textwidth}

      \begin{tikzpicture}[
        mem/.style={
          minimum width=2cm,
          inner sep=0pt,
          outer sep=0pt,
          draw=black
        },
        frame/.style={
          minimum width=2cm,
          inner sep=0pt,
          outer sep=0pt,
          draw=black,
          thick,
          fill=green!50!white
        },
        var/.style={
          node font=\ttfamily\scriptsize,
          minimum height=.5cm,
          minimum width=2cm,
          draw=black,
          fill=green!30!white,
          inner sep=0pt,
          outer sep=0pt
        },
        anchor=south west
        ]
        \node at (0,0) [
          mem,
          minimum height=6cm
        ] {};
        \visible<2-8>{
          \node (main) at (0,3) [
            frame,
            minimum height=2.5cm,
            label={[yshift=1.1cm]0:\scriptsize\tt main}
          ] {};
          \node (result) [
            var,
            above=3ex of main.south,
            label={180:\scriptsize\tt res}
          ] {\only<7-8>{12345}};
          \node (text) [
            var,
            minimum height=1cm,
            text width=2cm,
            align=center,
            above=0pt of result,
            "\scriptsize\tt text" left,
            "\scriptsize\tt\alert<4-7>{s}" {visible on=<4-7>,xshift=-.8cm,left}
          ] {\only<3->{\tiny\textit{Nel mezzo del cammin di nostra vita\ldots}}};
        }
        \visible<4-7>{
          \node (countwords) [
            frame,
            below=0pt of main,
            minimum height=1.7cm,
            label={[yshift=.5cm]0:\scriptsize\tt count_words}
          ] {};
          \node (count) [
            var,
            above=3ex of countwords.south,
            label={180:\scriptsize\tt count}
          ] {\only<5>{0}\only<6-7>{12345}};
          \node (sref) [
            var,
            above=0pt of count,
            label={180:\scriptsize\tt s$'$}
          ] {(*)};
        }
      \end{tikzpicture}
    \end{column}

  \end{columns}
\end{frame}

\begin{frame}[fragile]{\code{const} and references}

  \begin{codeblock}
std::string text\{\ddd\};
\uncover<2->{std::string\alert{\&} rtext\{text\};        // ok, can read/modify text via rtext}
\uncover<3->{std::string \alert{const\&} crtext\{text\}; // ok, crtext is a read-only view of text}\end{codeblock}

  \begin{codeblock}<4->{
std::string \alert{const} text\{\ddd\};
\uncover<5->{std::string\alert{\&} rtext\{text\};        // error, else could modify text via rtext}
\uncover<6->{std::string \alert{const\&} crtext\{text\}; // ok, can only read text via crtext}}\end{codeblock}

  \begin{codeblock}<7->{
int count_words(std::string \alert{const\&} s)
\{
  // this function can only read from s
   \ddd
\}}\end{codeblock}

\end{frame}

\begin{frame}[fragile]{\code{const} and pointers}

  \begin{codeblock}
std::string text\{\ddd\};
\uncover<2->{std::string\alert{*} ptext\{&text\};        // ok, can read/modify name via *ptext}
\uncover<3->{std::string \alert{const*} cptext\{&text\}; // ok, can only read text via *cptext}\end{codeblock}

  \begin{codeblock}<4->{
std::string \alert{const} text\{\ddd\};
\uncover<5->{std::string\alert{*} ptext\{&text\};        // error, else could modify text via *ptext}
\uncover<6->{std::string \alert{const*} cptext\{&text\}; // ok, can only read text via *cptext}}\end{codeblock}

\end{frame}

\begin{frame}[fragile]{How to pass arguments to functions}
  \begin{itemize}
  \item For input parameters
    \begin{itemize}
    \item<1-> If the type is primitive, pass by value
      \begin{codeblock}<.->{
int isqrt(int n);        // good
int isqrt(int const& n); // bad! pessimization}\end{codeblock}
    \item<2-> Otherwise pass by const reference
      \begin{codeblock}<.->{
int count\_words(std::string const\&);}\end{codeblock}
    \end{itemize}
  \item<3-> For output parameters, prefer to return a value or pass by non-const
    reference
    \begin{codeblock}
int read_from_cin();      // may be more difficult to overload for other types
void read_from_cin(int& n);\end{codeblock}
  \item<4-> For input-output parameters, pass by non-const reference
    \begin{codeblock}
void to_lowercase(std::string& s);\end{codeblock}
  \end{itemize}
\end{frame}

\begin{frame}[fragile]{Returning a reference}

  \begin{itemize}
  \item<1-> A function can return a reference only if the referenced object survives
    the end of the function
    \begin{itemize}[<.->]
    \item Otherwise, in the caller, the reference would refer to an object that
      doesn't exist anymore
    \end{itemize}
  \item<2-> In particular do not return a reference to a function local variable

    \begin{columns}
      \begin{column}{.5\textwidth}
        \begin{codeblock}
// bad
int& add(int a, int b) \{
  int result\{a + b\};
  return result;
\}\end{codeblock}
      \end{column}
      \begin{column}{.5\textwidth}
        \begin{codeblock}
// acceptable
int& increment(int& a) \{
  ++a;
  return a;
\}\end{codeblock}
      \end{column}
    \end{columns}

  \item<3-> Useful to chain multiple function calls on the same object
    \begin{codeblock}
std::string& to_lower(std::string& s) \{ \ddd ; return s; \}
std::string& trim_right(std::string& s) \{ \ddd ; return s; \}
std::string& trim_left(std::string& s) \{ \ddd ; return s; \}

std::string s\{\ddd\};
trim_left(trim_right(tolower(s)));\end{codeblock}

  \end{itemize}
  
\end{frame}

\begin{frame}[fragile]{\textit{range-for} loop}

    \code{for (} \textit{range-declaration} \code{:} \textit{range-expression} \code{)} \textit{statement}

  \begin{itemize}
  \item Simplified form of a \code{for} loop, to iterate on all the elements of
    a \textit{range} (sequence), such as a string of characters
  \item<2-> Execute repeatedly \textit{statement} for all the elements of the range
  \item<3-> \textit{range-declaration} declares a variable of the same type of
    an element of the range
    \begin{itemize}
    \item Can (and should) be a (const) reference
    \end{itemize}
  \item<4-> \textit{range-expression} represents the range to iterate over
  \item<5-> More on ranges later
  \end{itemize}

  \begin{codeblock}
std::string s\{"Hello!"\};
for (char& c : s) \{
  c = std::toupper(c); 
\}

for (int i : \{1, 2, 3, 4, 5\}) \{
  std::cout << i << \upquote{ };
\}\end{codeblock}

\end{frame}

\begin{frame}[fragile]{\code{auto}}

  Let the compiler deduce the type of a variable from the initializer, i.e. from
  the expression used to initialize the object

  \begin{codeblock}<2->{
auto z;                       \uncover<3->{// error, no initializer
auto i\{0\};                    }\uncover<4->{// int
auto const f\{0.F\};            }\uncover<5->{// float const
auto r\{i + f\};                }\uncover<6->{// float
auto s\{std::string\{"hello"\}\}; }\uncover<7->{// std::string
auto& rs\{s\};                  }\uncover<8->{// std::string&
auto const& cri\{i\};           }\uncover<9->{// int const&
auto j\{rs\};                   }\uncover<10->{// std::string
auto& g\{f\};                   }\uncover<11->{// float const&
auto p\{&i\};                   }\uncover<12->{// int*
auto q\{p\};                    }\uncover<13->{// int*
auto* t\{p\};                   }\uncover<14->{// int*}}\end{codeblock}

  \begin{itemize}
  \item<10-> \code{auto} never deduces a reference
  \item<11-> \code{auto} preserves constness of references
  \end{itemize}
\end{frame}

\begin{frame}[fragile]{\code{auto}/trailing return type}

  \begin{itemize}
  \item When the return type of a function is \code{auto} the compiler will
    deduce it from the return statement(s)
    \begin{itemize}
    \item Only usable in a function definition
    \end{itemize}

    \begin{codeblock}
auto isqrt(int n)    // auto deduced as int
\{
  int result;
  \ddd
  return result;
\}
\end{codeblock}

  \item<2-> If there are multiple return statements their expressions must have the
    same type
  \item<3-> Trailing return type
    \begin{codeblock}
int foo(\ddd);
auto foo(\ddd) -> int; // equivalent\end{codeblock}

  \end{itemize}

\end{frame}

\begin{frame}[fragile]{Enumerations}

  \begin{itemize}[<+->]
  \item An \textit{enumeration} is a distinct type with named constants, called
    \textit{enumerators}

  \begin{codeblock}
enum class Operator \{ Plus, Minus, Multiplies, Divides \};\end{codeblock}

  \item When used, an enumerator is specified with the name of the enumeration
    followed by the scope-resolution operator (\code{::})

    \begin{codeblock}
auto op\{Operator::Plus\}; // op is of type Operator\end{codeblock}

  \item By default, an enumerator has the value of the previous enumerator
    incremented by one and the first enumerator has value $0$
  \item An enumerator can be given a value explicitly
    \begin{codeblock}<.->{
enum class Operator \{ Plus = -2, Minus, Multiplies = 42, Divides \};}\end{codeblock}
    The values are respectively $-2$, $-1$, $42$ and $43$.

  \end{itemize}
\end{frame}

\begin{frame}[fragile]{Enumerations \insertcontinuationtext}

  \begin{itemize}[<+->]
  \item An enumeration has an integral \textit{underlying type}, by default \code{int}
  \item A different underlying type can be selected
    \begin{codeblock}<.->{
enum class byte : unsigned char \{ \};}\end{codeblock}

    Note that in this case there are no enumerators $\rightarrow$ a way to define a
    new integral type different from the underlying type

  \item All values of the underlying type are valid values for the enumeration
    object

    \begin{codeblock}<.->
Operator op\{55\};  // ok\end{codeblock}

  \item Conversions to the underlying type need to be explicit

    \begin{codeblock}<.->
int i\{Operator::Plus\};                     // error
auto i\{static_cast<int>(Operator::Plus)\};  // ok\end{codeblock}

  \end{itemize}

\end{frame}

\begin{frame}[fragile]{Enumerations \insertcontinuationtext}

  \begin{itemize}[<+->]
  \item There is also a less strict version of an enumeration: the
    \textit{unscoped} enumeration
    \begin{codeblock}
enum Operator \{ Plus, Minus, Multiplies, Divides \}; // NB no class\end{codeblock}

  \item The symbols of the enumerators are visible in the enclosing scope
    \begin{itemize}[<.->]
    \item i.e. the enumerators are not specified with the name of the
      enumeration followed by \code{::}
    \end{itemize}

    \begin{codeblock}<.->
Operator op\{Plus\};  // ok, no need to use Operator::
\end{codeblock}

  \item The conversion to the underlying type is implicit
    \begin{codeblock}<.->
int i\{Plus\};  // ok\end{codeblock}

  \item Prefer the scoped enumeration

  \end{itemize}

\end{frame}

\begin{frame}[fragile]{The \code{switch} statement}

  The \code{switch} statement transfers control to one of multiple statements,
  depending on the value of a condition

  \begin{codeblock}<2->{
double compute(char op, double left, double right)
\{
  double result;

  switch (op) \{
    case \upquote{+}:
      result = left + right;
      break;
    \ddd
    case \upquote{/}:
      result = (right != 0.) ? left / right : 0.;
      break;
    default:
      result = 0.;
  \}

  return result;
\}}\end{codeblock}

\end{frame}

\begin{frame}[fragile]{The \code{switch} statement \insertcontinuationtext}

  \begin{itemize}
  \item The condition is an expression whose evaluation gives an integral or
    enumeration value
    \begin{itemize}
    \item Cannot switch on strings, for example
    \end{itemize}

  \item An enumeration plays well with \code{switch} statements

    \begin{codeblock}
double compute(Operator op, double left, double right)
\{
  switch (op) \{
    case Operator::Plus:       return left + right;
    case Operator::Minus:      return left - right;
    case Operator::Multiplies: return left * right;
    case Operator::Divides:    return right != 0. ? left / right : 0.;
    default: return 0.;
  \}
\}\end{codeblock}

  \end{itemize}
\end{frame}

\begin{frame}{The \code{switch} statement \insertcontinuationtext}
  \begin{itemize}
  \item Each statement, possibly compound, is introduced either by a \code{case}
    label or by a \code{default} label
  \item Each \code{case} label specifies a unique integral constant
  \item Typically each statement is followed by a \code{break} statement, to
    jump after the \code{switch}, otherwise control \textit{falls through} the
    next instruction, even if this is part of a statement introduced by another
    label
    \begin{itemize}
    \item The compiler typically warns about falling through, but sometimes it's
      ok and the warning can be silenced with the \code{[[fallthrough]]}
      \textit{attribute}
    \end{itemize}
  \item There can be at most one \code{default} label, not necessarily at the
    end
  \item The same statement can be introduced by multiple \code{case} labels
  \end{itemize}

\end{frame}
