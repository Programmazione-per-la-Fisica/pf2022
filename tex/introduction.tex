
\section*{Introduction}

{
  \setbeamertemplate{footline}{}
  \setbeamertemplate{navigation symbols}{}
  \begin{frame}[plain,noframenumbering]
    \sectionpage
  \end{frame}
}

\begin{frame}{Collaboratori}

  \begin{itemize}
  \item Dott. Carlo Battilana, titolare modulo laboratorio A-L
  \item Dott. Fabio Ferrari, titolare modulo laboratorio M-Z
  \item Dott. Gianluca Bianco, tutor
  \item Dott. Giulio Colombini, tutor
  \item Dott. Simone Rossi Tisbeni, tutor
  \end{itemize}

\end{frame}

\begin{frame}{Orario primo semestre e ricevimento}
  \begin{itemize}
  \item Lezione/esercitazione: lunedì ore 11-13
  \item Laboratorio: indicativamente lunedì 14-18 (A-L) e martedì 9-13, su due turni ciascuno
    \begin{itemize}
    \item La partecipazione a Laboratorio è \textbf{obbligatoria}
    \end{itemize}
  \item Ricevimento su appuntamento, in presenza o via Teams
  \item Commenti, domande e suggerimenti sono benvenuti su chat (Teams) o via mail
  \end{itemize}
\end{frame}

\begin{frame}{Materiale di supporto}

  \begin{itemize}

  \item Presentazioni, guide, esercizi, \ldots a partire da
    \url{https://github.com/programmazione-per-la-fisica/}
    \begin{itemize}
    \item Materiale in parte replicato su Virtuale
    \end{itemize}

  \item Questa presentazione è distribuita in più modalità:
    \begin{itemize}
    \item un unico file \code{pdf} aggiornato incrementalmente a ogni lezione,
      disponibile a partire da\\
      {\smaller \url{https://github.com/programmazione-per-la-fisica/pf2022}},
      sia nella versione con animazioni sia nella versione
      \textit{print-friendly}
    \item più file \code{pdf} separati per ogni argomento, caricati su Virtuale,
      print-friendly
    \end{itemize}

  \end{itemize}

\end{frame}

\begin{frame}{Bibliografia consigliata}

  \begin{itemize}
  \item B.~Stroustrup, \href{https://stroustrup.com/tour3.html}{\textit{A tour
        of C++}}, 3rd edition (disponibile da ottobre 2022), Addison-Wesley.
    Parte dei contenuti della seconda edizione è disponibile online alla pagina
    \url{https://isocpp.org/tour}

  \item B.~Stroustrup,
    \href{https://stroustrup.com/programming.html}{\textit{Programming:
        Principles and Practice Using C++}}, 2nd edition, Addison-Wesley

  \item B.~Stroustrup, \href{https://stroustrup.com/4th.html}{\textit{The C++
        Programming Language}}, $4^{th}$ edition, Addison-Wesley

  \item B. Stroustrup, \textit{C++ -- Linguaggio, libreria standard, principi
      di programmazione}, IV edizione, Pearson

  \item Come referenza online: \Cpp{} reference, \url{https://cppreference.com/}

  \end{itemize}
\end{frame}

\begin{frame}{Modalità d'esame}

  L'esame consiste in due prove:

  \begin{enumerate}

  \item Progetto riguardante l'implementazione di un programma \Cpp{}. Il progetto
    è svolto in parte durante le ore di laboratorio, in parte in autonomia. E'
    raccomandato lo svolgimento in gruppo.

    Maggiori dettagli verso metà corso.

  \item Colloquio orale riguardante la discussione del progetto e domande
    teoriche e pratiche sugli argomenti svolti a lezione.

    Al colloquio si accede solo con una valutazione sufficiente del progetto.

  \end{enumerate}

\end{frame}

% sondaggio
\begin{frame}
  \begin{center}
    \vfill
    \url{https://forms.office.com/r/XznxKWh3wG}
    \vfill
    \includegraphics[height=.5\textheight]{images/sondaggio-qr.png}
    \vfill
  \end{center}
\end{frame}

\begin{frame}{Platforms, compilers, editors}
  \begin{itemize}[<+->]
  \item The reference platform is Linux (Ubuntu 22.04) with the \code{gcc}
    compiler suite
  \item But any platform with a recent compiler is fine, possibly with a
    Unix-like command-line shell (e.g. \code{bash} or \code{zsh})
  \item We'll provide some support to install and configure:
    \begin{itemize}[<.->]
    \item
      \href{https://github.com/giacomini/pf2021/blob/main/doc/WSLGuide.md}{Windows}:
      Ubuntu inside Window Subsystem for Linux
    \item
      \href{https://github.com/giacomini/pf2021/blob/main/doc/macOSGuide.md}{macOS}:
      XCode (-tools) with \code{gcc}
    \end{itemize}
  \item Any \textbf{textual} editor is fine
    \begin{itemize}[<.->]
    \item nano, vi, emacs, gedit, geany, notepad, \ldots
    \item we recommend \href{https://code.visualstudio.com/}{\textbf{Visual
          Studio Code}}
    \item LibreOffice Writer or MS Word are \textbf{not} text editors
    \end{itemize}
  \item You can also use an Integrated Development Environment (IDE)
    \begin{itemize}[<.->]
    \item Visual Studio, XCode, KDevelop, Eclipse, CLion, \ldots
    \end{itemize}
  \item Compilers online
    \begin{itemize}[<.->]
    \item \url{https://godbolt.org/}
    \item \url{https://repl.it/}
    \end{itemize}
  \end{itemize}

\end{frame}

\begin{frame}{Course outline}
  \begin{itemize}
  \item<1-> Introduction to Linux/Unix
  \item<2-> Elements of computer architecture and operating systems
  \item<3-> Why \Cpp{}
  \item<4-> Objects, types, variables
  \item<4-> Expressions
  \item<4-> Statements and structured programming
  \item<4-> Functions
  \item<4-> User-defined types and classes
  \item<4-> Generic programming and templates
  \item<4-> The Standard Library, containers, algorithms
  \item<4-> Dynamic memory allocation
  \item<4-> Dynamic polymorphism (aka object-oriented programming)
  \item<4-> Error management
  \item<5-> Elements of software engineering and supporting tools
  \end{itemize}
\end{frame}

\begin{frame}{Prossimi appuntamenti}
  \begin{itemize}
  \item \textbf{Iscrivetevi al corso su virtuale}
  \item Lunedì 10/10 per canale A-L, due turni 15-16.45 e 16.45-18.30, e martedì
    11/10 per canale M-Z, due turni 9-11 e 11-13, Aula II Irnerio
    \begin{itemize}
    \item Installazione/configurazione computer personali
    \item Con prenotazione
    \end{itemize}

  \item Lunedì 17/10 ore 11-13 lezione in Aula Magna Irnerio

  \item Lunedì 17/10 per canale A-L, due turni 15-16.45 e 16.45-18.30, e martedì
    18/10 per canale M-Z, due turni 9-11 e 11-13, Aula II Irnerio
    \begin{itemize}
    \item Introduzione a Linux/Unix
    \item Con prenotazione
    \end{itemize}

  \item Lunedì 24/10 ore 11-13 lezione in Aula Magna Irnerio

  \item Lunedì 24/10 per canale A-L, due turni 14-16 e 16-18, e martedì
    25/10 per canale M-Z, due turni 9-11 e 11-13, Aula II Irnerio
    \begin{itemize}
    \item Laboratorio
    \item Con prenotazione
    \end{itemize}

  \end{itemize}
\end{frame}
